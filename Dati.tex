%----------------------------------------------------------------------------------------
%   USEFUL COMMANDS
%----------------------------------------------------------------------------------------

\newcommand{\dipartimento}{Dipartimento di Matematica ``Tullio Levi-Civita''}

%----------------------------------------------------------------------------------------
% 	USER DATA
%----------------------------------------------------------------------------------------

% Data di approvazione del piano da parte del tutor interno; nel formato GG Mese AAAA
% compilare inserendo al posto di GG 2 cifre per il giorno, e al posto di 
% AAAA 4 cifre per l'anno
\newcommand{\dataApprovazione}{28 Febbraio 2022}

% Dati dello Studente
\newcommand{\nomeStudente}{Stefano}
\newcommand{\cognomeStudente}{Rizzo}
\newcommand{\matricolaStudente}{1193464}
\newcommand{\emailStudente}{stefano.rizzo.8@studenti.unipd.it}
\newcommand{\telStudente}{+ 39 342 78 19 582}

% Dati del Tutor Aziendale
\newcommand{\nomeTutorAziendale}{Luca}
\newcommand{\cognomeTutorAziendale}{Bizzaro}
\newcommand{\emailTutorAziendale}{luca.b@riskapp.it}
\newcommand{\telTutorAziendale}{+ 39 028 08 97 581}
\newcommand{\ruoloTutorAziendale}{}

% Dati dell'Azienda
\newcommand{\ragioneSocAzienda}{RiskApp S.r.l.}
\newcommand{\indirizzoAzienda}{Via Martiri della Libertà 19, 35026 Conselve PD}
\newcommand{\sitoAzienda}{https://www.riskapp.it/}
\newcommand{\emailAzienda}{info@riskapp.it}
\newcommand{\partitaIVAAzienda}{P.IVA IT04914960283}

% Dati del Tutor Interno (Docente)
\newcommand{\titoloTutorInterno}{Prof.}
\newcommand{\nomeTutorInterno}{Luigi}
\newcommand{\cognomeTutorInterno}{De Giovanni}

\newcommand{\prospettoSettimanale}{
     % Personalizzare indicando in lista, i vari task settimana per settimana
     % sostituire a XX il totale ore della settimana
    \begin{itemize}
        \item \textbf{Prima Settimana (40 ore)}
        \begin{itemize}
            \item Incontro con persone coinvolte nel progetto per discutere i requisiti e le richieste
            relativamente al sistema da sviluppare;
            \item Verifica credenziali e strumenti di lavoro assegnati;
            \item Presa visione dell’infrastruttura esistente;
            \item Inizio della formazione su React e Redux;
        \end{itemize}
        \item \textbf{Seconda Settimana - Formazione (40 ore)} 
        \begin{itemize}
            \item Studio dell'infrastruttura esistente;
            \item Studio di React e Redux;
        \end{itemize}
        \item \textbf{Terza Settimana - Studio e analisi (40 ore)} 
        \begin{itemize}
            \item Ricerca di eventuali nuove librerie in grado di apportare miglioramenti all'annotazione e visualizzazione dei file di tipo pdf;
            \item Studio delle nuove librerie;
            \item Studio di React e Redux;
        \end{itemize}
        \item \textbf{Quarta Settimana - Refactoring (40 ore)} 
        \begin{itemize}
            \item implementazione delle nuove librerie;
            \item revisione ed implementazione di modifiche alle componenti che gestiscono le annotazioni;
        \end{itemize}
        \item \textbf{Quinta Settimana - Refactoring (40 ore)} 
        \begin{itemize}
            \item implementazione delle modifiche alle componenti del frontend;
            \item definizione e scrittura dei test;
            \item produzione documentazione relativa alle modifiche apportate;
        \end{itemize}
        \item \textbf{Sesta Settimana - Analisi, progettazione e refactoring UI (40 ore)} 
        \begin{itemize}
            \item revisione della UI;
            \item produzione dei wireframe relativi alla nuova UI;
            \item inizio del refactoring dei componenti della UI;
        \end{itemize}
        \item \textbf{Settima Settimana - Refactoring e implementazione nuove componenti(40 ore)} 
        \begin{itemize}
            \item implementazione delle modifiche ai componenti della UI;
            \item progettazione e sviluppo dei componenti necessari per implementare la shortcut relativa alla funzionalità di annotazione;
        \end{itemize}
        \newpage
        \item \textbf{Ottava Settimana - Conclusione (40 ore)} 
        \begin{itemize}
            \item sviluppo dei test relativi ai componenti oggetto di refactoring o scritti ex-novo;
            \item produzione documentazione relative alle modifiche apportate alla UI ed ai componenti aggiunti;
            \item Collaudo e verifica del lavoro prodotto;
        \end{itemize}
    \end{itemize}
}

% Indicare il totale complessivo (deve essere compreso tra le 300 e le 320 ore)
\newcommand{\totaleOre}{320}

\newcommand{\obiettiviObbligatori}{
	 \item \underline{\textit{O01}}: Deve essere eseguito il refactoring dei componenti che gestiscono le annotazioni;
	 \item \underline{\textit{O02}}: Deve essere rivista la collocazione e l'aspetto visivo delle toolbar per la navigazione del pdf con le varie annotazioni;
	 
}

\newcommand{\obiettiviDesiderabili}{
	 \item \underline{\textit{D01}}: Deve essere aggiunta una shortcut per l'attivazione della funzionalità di annotazione del pdf;
}

\newcommand{\obiettiviFacoltativi}{
	 \item \underline{\textit{F01}}: devono essre presenti dei test a livello di frontend;
}